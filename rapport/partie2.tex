\newpage
\section{Extension de l’application}
\subsection{Structure d'un fichier grilleXX.txt}
Ce fichier commence  par les dimensions de la futur grille d’abord le nombre de région par 
largeur puis un espace ensuite le nombre de région par hauteur puis saut de ligne.
Le fichier continue par les valeurs des cases fixes espacées d’un espace entre chaque valeur. 
Les cases modifiables sont représentées par un zéro. Chaque ligne du fichier représente une 
ligne du sudoku comme suit la figure ci-dessous.
\begin{figure}[ht]
  \caption{\label{annexe1} fichier text}
  \includegraphics [width=80mm]{images/Fichier_text.png} \\[0.5cm]
\end{figure}

Le modèle transforme le fichier texte figure6 en cette grille figure7\\

\begin{figure}[ht]
  \caption{\label{annexe1} grille modélisé}
  \includegraphics [width=80mm]{images/fichier_type.png} \\[0.5cm]
\end{figure}
\newpage
\subsection{Création d’une règle}
On peut ajouter de nouvelle règle à l’application, pour celà il faut générer un rapport 
c’est-à-dire être une sous-classe de ReportGenerator.\\
Une règle est constituée de sa description, d’une liste de valeur (ajout d’une valeur dans 
une case ou suppressions des candidats dans une ou plusieurs case(s) ) et de quatre ensembles 
(pour la surbrillance de l’aide) : ensemble des cellules concernées par la suppression d’un 
candidat les cases (rouge foncé) DELETION\_CELLS, leurs unités (rouge clair) DELETION\_UNITS et 
l’ensemble des cellules qui se base sur le raisonnement de la règle les cases (bleu foncé) 
DECISIVE\_UNITS et leurs unités (bleu clair) DECISIVE\_UNITS.\\\\
Si DELETION\_CELLS n’est pas vide on peut en conclut que l’action de la règle est de supprimer 
les candidats de la liste dans cet ensemble de cellules. Au contraire si cet ensemble est vide 
c’est qu’on doit ajouter la valeur  dans DECISIVE\_CELLS.\\
Si la règle n’aboutit à aucune avancer dans la grille (suppression candidat ou ajout valeur) 
elle renvoit null.


\newpage
\section{Interface}

Notre application se présente de la manière suivante :\\
une barre de menu situé au nord de la fenêtre vous permettra de choisir 
une action à réaliser parmi toutes les fonctionnalités qui vous sont proposés.\\
\\
La grille de sudoku occupe le centre de notre application et se retrouve accompagnée 
d'une colonne, à l'est de la fenêtre, vous permettant de sélectionner 
parmi des boutons raccourcis, les fonctionnalités les plus utilisés tel défaire, 
refaire une action, demander un indice, réinitialiser la grille etc, puis, 
une zone d'aide, situé en bas de la grille.\\

Il vous est également possible d'utiliser votre clavier afin de profiter 
des fonctionnalités que vous offre notre application.\\
Pour ce faire, il vous suffit de saisir la combinaison appropriée, combinaison 
que vous pourrez retrouver en consultant les listes des fonctionnalités disponibles dans le menu, 
au nord de l'application, où coïncide chaque fonctionnalité avec sa combinaison 
de la forme CTRL+LETTRE, qu'il vous suffit d'utiliser en appuyant simultanément 
sur la touche contrôle et la lettre correspondante à l'action souhaîtée. \\

Pour ajouter/retirer des possibilités dans la grille, il vous suffit d'utiliser 
le bouton droit de la souris tandis que pour ajouter/retirer des candidats, il 
faudra utiliser le bouton gauche de votre souris.

\begin{figure}[ht]
  \caption{\label{annexe6} interface sudoku}
  \includegraphics [width=130mm]{images/interface.png} \\[0.5cm]
\end{figure}

\newpage
\section{Jouabilité}
Dans le menu fichier, vous pourrez choisir entre choisir 
une nouvelle grille (par défault, la grille numéro 2 est choisie),
ouvrir un fichier de sauvegarde, 
sauvegarder la partie ou encore quitter la partie.

\begin{figure}[ht]
  \caption{\label{annexe7} menu fichier}
  \includegraphics [width=130mm]{images/fichier.png} \\[0.5cm]
\end{figure}

\newpage
Dans le menu édition, vous aurez la possibilité de réinitialiser la grille, 
c'est-à-dire remettre votre grille choisie dans son état initial, 
résoudre pas à pas qui consiste à remplir les cases vides de la grille
en expliquant comment procéder grâce à un petit texte apparaissant en bas de l'écran, 
l'indice, donne un message d'aide et surligne la ligne correspondant d'une couleur bleu 
sans donner la réponse, demander la solution complète de la grille, 
refaire l'action et défaire l'action précédente.

\begin{figure}[ht]
  \caption{\label{annexe8} menu édition}
  \includegraphics [width=130mm]{images/edition.png} \\[0.5cm]
\end{figure}

\newpage
Dans le menu aide, vous aurez la possibilité de consulter 
les règles du jeu et un guide d'utilisation de notre application.

\begin{figure}[ht]
  \caption{\label{annexe9} menu aide}
  \includegraphics [width=130mm]{images/aide.png} \\[0.5cm]
\end{figure}

\newpage
Toutes ces opérations sont également accessibles grâce aux raccourcis clavier suivant :
\begin{itemize}
 \item CTRL+N : nouvelle grille
 \item CTRL+O : ouvrir un fichier de sauvegarde
 \item CTRL+S : sauvegarder sa partie
 \item CTRL+Q : quitter le jeu
 \item CTRL+R : réinitialiser
 \item CTRL+P : résoudre pas à pas
 \item CTRL+C : avoir un indice
 \item CTRL+F : solution complète
 \item CTRL+Y : refaire l'action
 \item CTRL+Z : annuler l'action
 \item CTRL+T : tutoriel
 \item CTRL+G : comment jouer
\end{itemize}

\newpage
Lorsque vous appuyez sur le bouton pause, une fenêtre apparaît 
et vous propose de reprendre la partie ou de la quitter.
\begin{figure}[ht]
  \caption{\label{annexe10} pause}
  \includegraphics [width=70mm]{images/pause.png} \\[0.5cm]
\end{figure}

\newpage
Si vous appuyez sur le bouton solution, la grille sera rempli entièrement.
\begin{figure}[ht]
  \caption{\label{annexe11} solution complète}
  \includegraphics [width=130mm]{images/solution.png} \\[0.5cm]
\end{figure}

\newpage
Vous avez la possibilité de réinitialiser la grille de sudoku 
et ainsi de recommencer la partie. 
\begin{figure}[ht]
  \caption{\label{annexe12} réinitialisation}
  \includegraphics [width=130mm]{images/reinit.png} \\[0.5cm]
\end{figure}

\newpage
Si vous demandez un indice, vous pourrez avoir un texte d'aide 
et pourrez compléter la grille en vous aidant de la surbrillance 
des cases concernés.  
\begin{figure}[ht]
  \caption{\label{annexe13} indice}
  \includegraphics [width=130mm]{images/clue.png} \\[0.5cm]
\end{figure}

\newpage
En choisissantla méthode pas à pas, vous bénéficierez à la fois de conseils
pour remplir la grille mais également de la réponse de la case concerné 
vous permettant de résoudre la grille de sudoku. 
\begin{figure}[ht]
  \caption{\label{annexe14} pas à pas}
  \includegraphics [width=130mm]{images/pas.png} \\[0.5cm]
\end{figure}

\newpage
Il vous est posssible de demander une nouvelle grille, en chargeant 
un fichier grilleXX.txt. L'application supporte différents formats de grilles. 
\begin{figure}[ht]
  \caption{\label{annexe15} nouvelle grille}
  \includegraphics [width=130mm]{images/newGrid.png} \\[0.5cm]
\end{figure}

\newpage
\section{Difficultés rencontrées}
Lors de la conception de notre projet, nous avons rencontrés quelques difficultés :
\begin{itemize}
 \item gestion du temps, où chaque semaine, nous nous fixions des objectifs à atteindre,
 \item utilisation et prise en main du logiciel git et du site 
	    \href{https://github.com/yuki1996/Sudoku}{github.com}, 
 \item respect des demandes de la part du client, 
 \item incompatibilité/réajustement lors de réunion de différents travaux des membres du groupe
\end{itemize}
\section{Répartition du travail}
Dans un premier temps, nous avons tous ensemble commencer à réfléchir à l'architecture du modèle.
Puis au moment quand on a eu une structure claire, Loïc, Paul et Laëtitia ont codé le modèle quant
à Florian, il a commencé à concevoir et élaborer le design et les fonctionnements de l'application du jeu.
Quant le modèle fut fini et la réflexion sur les structure des règles et du gestionnaire d'heuristiques
achevée Laëtitia et Paul se sont mis à les coder y compris l'historique. Quant à Loïc, il est parti avec 
Florian sur la partie graphique du projet mais plus centré sur les composants grille et cellule (surbrillance...)
Pour finir, chacun a donné un coup de main dans toutes les parties du projet pour régler les derniers beugs et erreurs.

\newpage
\section{Conclusion}
Ce projet nous a apporté une grande expérience car il s'agit 
de notre premier ``gros'' projet en équipe de plus de deux personnes.

Afin de réaliser un travail commun, nous avons opté pour un service 
de gestion de développement de logiciels, utilisant le logiciel 
de gestion de versions Git ainsi que le service web d'hébergement \url{https://github.com/}.

Ce dernier fut pour nous d'une grande aide, ce fut une toute nouvelle façon de 
procéder que nous a donné comme opportunité ce projet même si la prise en main 
fut assez compliqué.

Pour finir, ce projet fut pour nous très enrichissant car il nous a permi de réunir
l'ensemble de nos connaissances au sein d'un même projet. Nous avons fait le choix 
de développer notre application dans le langage de programmation Java, qui est un langage
très intéressant pour les développements d'applications graphiques.

\begin{thebibliography}{9}
 \bibitem{Site Internet}
          \url{https://www.mots-croises.ch/Sudoku.htm}
 \bibitem{Site Internet}
          \url{http://www.le-sudoku.fr/}
\end{thebibliography}

\section{Annexes}
\begin{figure}[ht]
  \caption{\label{annexe16} Diagramme model}
  \includegraphics [width=160mm]{images/model.png} \\[0.5cm]
\end{figure}

\begin{figure}[ht]
  \caption{\label{annexe17} Diagramme history}
  \includegraphics [width=140mm]{images/history.png} \\[0.5cm]
\end{figure}

\begin{figure}[ht]
  \caption{\label{annexe18} Diagramme heuristic}
  \includegraphics [width=140mm]{images/heuristic.png} \\[0.5cm]
\end{figure}



